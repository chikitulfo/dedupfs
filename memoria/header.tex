%%Packages and settings I use in most documents.
\usepackage[spanish]{babel}
\usepackage{ucs}
\usepackage[utf8x]{inputenc}
%\usepackage[T1]{fontenc} %Produce una salida llena de caracteres raros
\usepackage{xcolor}
\usepackage{fancyvrb}

\usepackage[section]{placeins}

\usepackage{amsmath,amsfonts}
\usepackage{graphicx}
\usepackage[space]{grffile} %% Para incluir archivos con espacios
\usepackage{hyperref}
\hypersetup{
    colorlinks=false,
    citecolor=black,
    filecolor=black,
    linkcolor=black,
    urlcolor=black,
    linkbordercolor=gray!70
}
\usepackage[all]{hypcap}    %for going to the top of an image when a figure reference is clicked

\usepackage{fancyhdr} 	%%Para encabezados

\usepackage{parcolumns}
\usepackage{listings}
\lstset{
  literate={ö}{{\"o}}1
           {ä}{{\"a}}1
           {ü}{{\"u}}1
           {á}{{\'a}}1
           {Á}{{\'A}}1
           {é}{{\'e}}1
           {É}{{\'E}}1
           {í}{{\'i}}1
           {Í}{{\'I}}1
           {ó}{{\'o}}1
           {Ó}{{\'O}}1
           {ú}{{\'u}}1
           {Ú}{{\'U}}1
           {ñ}{{\~{n}}}1
           {Ñ}{{\~{N}}}1
           {ç}{{\c{c}}}1
           {Ç}{{\c{C}}}1
}
\usepackage{tcolorbox} %%para utilizar cajas que no se quedan abiertas en los listings
\tcbuselibrary{listings,skins,breakable}
\newtcblisting{mycode}{
      arc=0mm,
      top=0mm,
      bottom=0mm,
      left=3mm,
      right=0mm,
      width=\textwidth,
      boxrule=1pt,
      colback=blue!10,
      listing only,
      listing options={style=mystyle},
      breakable
}
%% Mi propio tcb input
\newcommand{\mytcbinputlisting}[1]
{
  \tcbinputlisting{  
    listing file=#1,
    arc=0mm,
    top=0mm,
    bottom=0mm,
    left=3mm,
    right=0mm,
    width=\textwidth,
    boxrule=1pt,
    colback=blue!10,
    listing only,
    listing options={style=mystyle},
    breakable
  }
}
\lstdefinestyle{mystyle}{
  language=C,
  basicstyle=\footnotesize\ttfamily,
  keywordstyle=\color{blue},
  stringstyle=\color{red},
  commentstyle=\color{black!75},
  directivestyle={\color{magenta}},
  breakatwhitespace=false,
  showstringspaces = false,
  numbers = left,
  numbersep = 15pt,
  numberstyle = \footnotesize,
  stepnumber=5,
  breaklines=true,
  tabsize=2
}

\textheight=22.94cm \textwidth=17cm \topmargin=-1cm
\oddsidemargin=-0.4cm
\parindent=1.0cm

%Para permitir figuras más a menudo:
\renewcommand{\topfraction}{.85}
\renewcommand{\bottomfraction}{.7}
\renewcommand{\textfraction}{.15}
\renewcommand{\floatpagefraction}{.66}
\renewcommand{\dbltopfraction}{.66}
\renewcommand{\dblfloatpagefraction}{.66}
\setcounter{topnumber}{9}
\setcounter{bottomnumber}{9}
\setcounter{totalnumber}{20}
\setcounter{dbltopnumber}{9}

\setlength{\parskip}{\baselineskip} %Esto se utiliza para echarle cuenta a los intros

\newcommand{\encabezados}[2]{
  \pagestyle{fancy}
  \renewcommand{\headrulewidth}{0.4pt}
  \renewcommand{\footrulewidth}{0.0pt}
  \fancyhf{}
  \fancyhead[LO]{\leftmark}
  \fancyhead[RE]{\rightmark}
  \fancyhead[CE]{}
  \fancyhead[RO]{#1}
  \fancyfoot[RO]{\thepage}
  \fancyfoot[LO]{#2}
}

% Para guardar el número de un enumerate y continuar después
% \newcounter{saveenumi}
% \newcommand{\savecounteri}{\setcounter{saveenumi}{\value{enumi}}}
% \newcommand{\restorecounteri}{\setcounter{enumi}{\value{saveenumi}}}
