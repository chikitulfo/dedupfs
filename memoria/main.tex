\documentclass[12pt,a4paper]{article}

%%Packages and settings I use in most documents.
\usepackage[spanish]{babel}
\usepackage{ucs}
\usepackage[utf8x]{inputenc}
%\usepackage[T1]{fontenc} %Produce una salida llena de caracteres raros
\usepackage{xcolor}
\usepackage{fancyvrb}

\usepackage[section]{placeins}

\usepackage{amsmath,amsfonts}
\usepackage{graphicx}
\usepackage[space]{grffile} %% Para incluir archivos con espacios
\usepackage{hyperref}
\hypersetup{
    colorlinks=false,
    citecolor=black,
    filecolor=black,
    linkcolor=black,
    urlcolor=black,
    linkbordercolor=gray
}
\usepackage[all]{hypcap}    %for going to the top of an image when a figure reference is clicked

\usepackage{fancyhdr} 	%%Para encabezados

\usepackage{parcolumns}
\usepackage{listings}
\lstset{
  literate={ö}{{\"o}}1
           {ä}{{\"a}}1
           {ü}{{\"u}}1
           {á}{{\'a}}1
           {Á}{{\'A}}1
           {é}{{\'e}}1
           {É}{{\'E}}1
           {í}{{\'i}}1
           {Í}{{\'I}}1
           {ó}{{\'o}}1
           {Ó}{{\'O}}1
           {ú}{{\'u}}1
           {Ú}{{\'U}}1
           {ñ}{{\~{n}}}1
           {Ñ}{{\~{N}}}1
           {ç}{{\c{c}}}1
           {Ç}{{\c{C}}}1
}
\usepackage{tcolorbox} %%para utilizar cajas que no se quedan abiertas en los listings
\tcbuselibrary{listings,skins,breakable}
\newtcblisting{mycode}{
      arc=0mm,
      top=0mm,
      bottom=0mm,
      left=3mm,
      right=0mm,
      width=\textwidth,
      boxrule=1pt,
      colback=blue!10,
      listing only,
      listing options={style=mystyle},
      breakable
}
%% Mi propio tcb input
\newcommand{\mytcbinputlisting}[1]
{
  \tcbinputlisting{  
    listing file=#1,
    arc=0mm,
    top=0mm,
    bottom=0mm,
    left=3mm,
    right=0mm,
    width=\textwidth,
    boxrule=1pt,
    colback=blue!10,
    listing only,
    listing options={style=mystyle},
    breakable
  }
}
\lstdefinestyle{mystyle}{
  language=C,
  basicstyle=\footnotesize\ttfamily,
  keywordstyle=\color{blue},
  stringstyle=\color{red},
  commentstyle=\color{black!75},
  directivestyle={\color{magenta}},
  breakatwhitespace=false,
  showstringspaces = false,
  numbers = left,
  numbersep = 15pt,
  numberstyle = \footnotesize,
  stepnumber=5,
  breaklines=true,
  tabsize=2
}

\textheight=22.94cm \textwidth=17cm \topmargin=-1cm
\oddsidemargin=-0.4cm
\parindent=1.0cm

%Para permitir figuras más a menudo:
\renewcommand{\topfraction}{.85}
\renewcommand{\bottomfraction}{.7}
\renewcommand{\textfraction}{.15}
\renewcommand{\floatpagefraction}{.66}
\renewcommand{\dbltopfraction}{.66}
\renewcommand{\dblfloatpagefraction}{.66}
\setcounter{topnumber}{9}
\setcounter{bottomnumber}{9}
\setcounter{totalnumber}{20}
\setcounter{dbltopnumber}{9}

\setlength{\parskip}{\baselineskip} %Esto se utiliza para echarle cuenta a los intros

\newcommand{\encabezados}[2]{
  \pagestyle{fancy}
  \renewcommand{\headrulewidth}{0.4pt}
  \renewcommand{\footrulewidth}{0.0pt}
  \fancyhf{}
  \fancyhead[LO]{\leftmark}
  \fancyhead[RE]{\rightmark}
  \fancyhead[CE]{}
  \fancyhead[RO]{#1}
  \fancyfoot[RO]{\thepage}
  \fancyfoot[LO]{#2}
}

% Para guardar el número de un enumerate y continuar después
% \newcounter{saveenumi}
% \newcommand{\savecounteri}{\setcounter{saveenumi}{\value{enumi}}}
% \newcommand{\restorecounteri}{\setcounter{enumi}{\value{saveenumi}}}

%%%%%%%%%%%%%%%%%%%%%%%%%%%%%%
%% Definición de la Portada %%
%%%%%%%%%%%%%%%%%%%%%%%%%%%%%%
%%PARÁMETROS
%%1: Nombre de la Asignatura
%%2: Nombre del Profesor
%%3: Curso
%%4: Título
%%5: Subtítulo
\newcommand{\portada}[5]{
\begin{titlepage}
  \begin{tabular}{l}
    \sc Universidad de Murcia \\
    \sc Facultad de Informática \\
	\sc #1\\
	\sc #2\\
	\sc #3
  \vspace*{1.9cm}\mbox{}
  \end{tabular}
  \bigskip

  \vspace*{0.5 cm}
  \begin{center}
  \textbf{\Huge #4 } \\ [1.0cm]
  \textbf{\Large #5}\\
  \end{center}
  \vspace*{9 cm}
  \begin{flushright}
    \begin{tabular}{ll}
		Juan José Andreu Blázquez\\ $<$juanjose.andreu@um.es$>$ \\
         \today \\
    \end{tabular}
  \end{flushright}
\end{titlepage}
}



\begin{document}
\portada{Diseño y Estructura Interna de un SO}{Juan Piernas, Diego Sevilla}{5º de Ingeniería Informática}{Dedupfs\\\vspace{0.4cm} Sistema de ficheros con deduplicación basado en FUSE}{Convocatoria de Junio, curso 2014-2015}

\tableofcontents
\newpage
\section{Resumen del trabajo realizado}

El objetivo de este proyecto de programación práctico ha sido desarrollar un sistema de ficheros capaz de detectar cuándo se almacenan ficheros idénticos, y dejar solo una copia de los datos en el sistema de ficheros, haciendo que el resto de ficheros sean simples apuntadores a la copia que contiene los datos. Esto permite que, en sistemas de ficheros donde hay muchos ficheros iguales, se ahorre una cantidad importante de espacio en disco.

Esta deduplicación de ficheros idénticos, se hace de forma transparente al usuario del sistema de ficheros, por lo que se siguen viendo ambos archivos como archivos diferentes, y en caso de modificarse uno de ellos, se volverán a separar y quedarán como dos ficheros diferentes. El usuario no llega a percibir estos cambios, y él utiliza su sistema de ficheros de forma normal, pero se ahorra espacio si tiene muchos archivos iguales.
Para detectar si dos ficheros son idénticos, se ha utilizado la función SHA1 para calcular la clave de dispersión del archivo. Con esta clave se almacenan los datos en un directorio del sf, con un fichero que tiene como nombre el código hexadecimal del hash, y se deja el fichero original vacío como un marcador. En una base de datos interna, se almacenan una relación de cada nombre de fichero y qué clave de dispersión le corresponde, además de algunos parámetros adicionales que se indican más adelante. Esto permite saber dónde se encuentran realmente los datos de cada fichero, y se pueden mover fácilmente de un lugar a otro cuando se modifican y se recalcula su clave de dispersión.

El cálculo de la clave de dispersión es algo costoso, puesto que implica leer el fichero completo, y realizar cálculos sobre los datos del fichero para obtener la clave. Debido a esto, es necesario buscar el momento más adecuado para calcular la clave y evitar un impacto considerable sobre la velocidad del sistema de ficheros. Este momento adecuado se produce cuando un archivo ha sido abierto para escribir en él por uno o más procesos, se ha escrito en él, y lo cierra el último de los procesos que lo tenía abierto. Entonces se recalcula la clave de dispersión y, si ésta ha cambiado, se mueven los datos al fichero que corresponda a esa clave.

Además del deduplicado de archivos, también es necesario asegurarse de que otro tipo de funcionalidades típicas de un SF en UNIX están correctamente implementadas. Para asegurar que el funcionamiento de los enlaces físicos permanece intacto en nuestro SF, se ha utilizado una tabla adicional en la base de datos que almacena para cada enlace, cuál es el archivo que posee los contenidos. En cuanto a los enlaces simbólicos, no es necesario llevar a cabo ninguna acción especial para que funcionen de forma normal.

Por último, los permisos y el resto de atributos que tiene un archivo, se almacenan en el archivo vacío que se deja de marcador, a excepción del tamaño de archivo, que también se almacena en la base de datos junto a cada archivo para facilitar el acceso posterior a esta información.

\newpage
\section{Glosario de conceptos}

A lo largo de esta memoria explicativa del desarrollo de la práctica propuesta, se utilizan una serie de términos cuyo significado se explica a continuación:
\begin{itemize}
 \item \textbf{Hash/Clave de dispersión}: Clave de 20 bytes generada a partir de un archivo mediante la función criptográfica SHA1 y que se utiliza como identificador del contenido de un archivo.
 \item \textbf{Deduplicar/deduplicado}: Cuando se habla de deduplicar un archivo, o de uno deduplicado, se hace referencia a la acción de coger dos archivos idénticos que están duplicados en el sf, y dejar solo uno conteniendo los datos.
 \item \textbf{Marcador}: Un marcador es el archivo vacío con el nombre original que se deja en el lugar donde se creó un fichero en primer lugar.
 \item \textbf{Archivo de datos}: Archivo que se crea en una carpeta oculta del sistema de ficheros y que contiene los datos a los que apuntan los marcadores. Su nombre se corresponde con el hash generado por los datos contenidos en él.
 \item \textbf{Reduplicar}: Este término se refiere a un archivo que se vuelve a duplicar cuando estaba deduplicado, porque sus contenidos han cambiado.
 \item \textbf{Punto de montaje}: Es el punto del árbol de ficheros en el que se monta el sistema de ficheros dedupfs, y desde donde se interactúa con él. En este directorio no se almacena información, si no que sirve de lugar de acceso a los datos.
 \item \textbf{Directorio subyacente/raíz}: Corresponde con el subárbol de ficheros donde se almacenan realmente los datos del sistema de ficheros dedupfs. Si se accede a los datos a través de este directorio, en puesto de a través del punto de montaje, sólo se verán los marcadores vacíos, y no se podrá acceder a sus datos.
\end{itemize}


\newpage
\section{Descripción del sistema de ficheros}

El diseño e implementación del sistema de ficheros \emph{Dedupfs} se ha llevado a cabo teniendo en cuenta desde el primer momento cuáles eran los principales retos a tener en cuenta para lograr la deduplicación totalmente transparente al usuario.
\begin{itemize}
 \item Es necesario conocer qué ficheros están deduplicados y cuáles no, y dónde se encuentran sus datos, para ello se utiliza una base de datos sqlite que se almacena en la dirección \texttt{\small .dedupfs/dedupfs.db}.
 \item Es necesario que los archivos deduplicados permanezcan independientes de cara al usuario del sistema de ficheros, y que los camios producidos en uno de ellos no se propaguen a otros. Además, sus permisos, tiempos de modificación, y otros atributos, deben permanecer independientes. Por este motivo se ha optado por dejar un marcador vacío en el lugar de cada archivo, donde se conservan sus permisos y otros atributos, y almacenar los datos en una subcarpeta del directorio \texttt{\small .dedupfs/data} con el hash de los datos como nombre de archivo.
 \item Los enlaces físicos sirven para designar al mismo fichero con diferentes nombres desde el punto de vista del usuario. Es necesario tratarlo de la forma adecuada, y evita tratar enlaces físicos como si fueran archivos deduplicados. Para ello se ha creado una segunda tabla en la base de datos, que contabiliza estos enlaces y permite gestionar los archivos.
 \item Para evitar realizar el cálculo de hash cuando un archivo todavía puede ser modificado, lo más adecuado es hacerlo solamente cuando un fichero haya sido modificado, y deja de ser utilizado. Para ello se almacena en memoria un mapa de archivos abiertos para escritura, se marca si son modificados, y se computa su hash cuando los cierra el último proceso que los tenga abiertos.
\end{itemize}

Todo lo anterior se ha implementado modificando las funciones principales del ejemplo \texttt{\small bbfs.c}, copiándolas en otro llamado \texttt{\small dedupfs.c}, y añadiendo algunas nuevas.

También, se ha incluido un módulo llamado \texttt{\small database} que se encarga de gestionar toda la comunicación del módulo principal con las distintas tablas de la base de datos y con el mapa de ficheros abiertos (que también ha sido implementado como una base de datos sqlite en memoria).

\subsection{Estructuras de datos}

Como se ha indicado en el apartado anterior, las estructuras de datos utilizadas en el proyecto se han basado en bases de datos sqlite3, tanto para el almacenamiento de la información de archivos deduplicados y sus hashes, como para el mapa de archivos abiertos en memoria. Incluyendo también los enlaces físicos.

\begin{itemize}
 \item \textbf{Base de datos de archivos}: La base de datos de archivos está formada por dos tablas, y se almacena en \texttt{\small .dedupfs/dedupfs.db} dentro del directorio subyacente. La primera tabla, la tabla de archivos \emph{(files)}, contiene una entrada por cada archivo de datos almacenado en nuestro SF. En dicha entrada se almacena la siguiente información:
  \begin{itemize}
    \item Path: El path del archivo lógico. Que se dejará como marcador.
    \item Shasum: El hash SHA1 de los datos del archivo, almacenado en texto con representación hexadecimal.
    \item Datapath: El path donde se almacenan físicamente los datos del archivo en el SF.
    \item Size: El tamaño en bytes del archivo, para acceder a esta información cómodamente, puesto que no está disponible en le marcador.
    \item Deduplicados: La cuenta de veces que están deduplicados los datos del archivo, empieza en 0 para archivos no deduplicados.
  \end{itemize}
  Con esta información se pueden realizar fácilmente todas las operaciones necesarias para deduplicar y reduplicar los archivos, y saber en qué estado está cada uno en cada momento.\\
  
  En esta base de datos de datos también se guarda la tabla de enlaces físicos, llamada \emph{links}. En ella se guarda para cada enlace físico, cuál es el archivo que se debe buscar en la tabla de archivos para encontrar sus datos.
    \begin{itemize}
    \item Linkpath: El path del enlace físico.
    \item Originalpath: El path del archivo cuya entrada de la tabla de archivos posee toda la información.
  \end{itemize}
  
  \item \textbf{Mapa de ficheros abiertos}: En esta estructura se guarda qué archivos han sido abiertos para escritura, de forma que se pueda saber cuando se cierran, si han sido modificados y si es el último descriptor siendo cerrado de ese archivo. Este mapa se ha establecido como una base de datos sqlite residente en memoria, ya que solo es necesario durante la ejecución del SF. Cada entrada de la tabla contiene la siguiente información:
  \begin{itemize}
   \item fh: Descriptor de ficheros utilizado para abrir este archivo.
   \item path: Path del archivo lógico que está abierto para escritura.
   \item deduplicado: Si el archivo está deduplicado o no.
   \item modificado: Indicador de si el archivo ha sido modificado o no. Un archivo puede abrirse para escritura, y finalmente no ser modificado.
  \end{itemize}
\end{itemize}

Con estas tres tablas, dedupfs tiene toda la información necesaria de cada archivo para realizar cualquier operación que se le solicite.

\subsection{Implementación de la funcionalidad}

La funcionalidad completa del sistema de ficheros dedupfs dista bastante de ser trivial en muchas de las funciones que ha habido que reimplementar. Para explicar el funcionamiento de las nuevas funciones implementadas, vamos a seguir el flujo de ejeución, explicando cada una de las partes implicadas en él.

\subsubsection{Apertura, lectura, escritura, y cierre de un fichero}

Estas cuatro operaciones componen la funcionalidad fundamental del sistema de ficheros. Pasamos a explicar el comportamiento de dedupfs en estas operaciones.

\paragraph{Apertura:}
Cuando se produce la apertura de un archivo, dedupfs recibe una llamada a la función \textbf{bb\_open} con el path del archivo, y los indicadores de apertura correspondientes. Lo primero que hace el SF en este caso, es comprobar que puede abrir el marcador. Si no hay ningún problema al acceder al marcador, entonces se abre el archivo de datos: Se mira si es un enlace duro, y de serlo, se obtiene el archivo al que apunta. Tras esto, se busca este archivo en la base de datos. En caso de estar en la bd, se coge el path de su archivo de datos. Si está deduplicado y se abre en escritura, es necesario abrir una copia, para evitar contaminar el archivo de datos deduplicado (se realiza esta copia si es el primer open de este archivo), esta copia se llama idéntica al archivo de datos, sólo que añadiendo una 'w' al final.
Si no estuviera en la base de datos, simplemente se utilizará el marcador.
Finalmente se abre el path que corresponda, y una vez abierto y si se hace en escritura, se introduce en el mapa de archivos abiertos para escritura.

La creación de un archivo nuevo puede verse como un caso particular de apertura en el que se abre un archivo nuevo para escribir en él. En este caso, cuando se recibe una llamada a \textbf{bb\_create}, simplemente se abre el archivo y se introduce en el mapa de archivos abiertos.


\newpage
\section{Manual de uso}

\newpage
\section{Pruebas del sistema de ficheros}

%% Linux kernel sources 4.0.4
% rootdir: 97609 archivos. du: 675M
% nondeup: 48948 archivos. du: 648M

\newpage
\section{Anexo: Listado del código fuente}

\iffalse 
\begin{figure}[h!]
  \centering
  \label{fig:supervisiontree}
  \includegraphics[width=.95\linewidth]{supervisiontree}
  \caption{Árbol de supervisión}
\end{figure}
\fi


\iffalse
\lstdefinestyle{mystyle}{language=Pascal,
  breakatwhitespace=false,
  breaklines=true,
  basicstyle=\footnotesize
}
\begin{mycode}
1> application:start(otpapp).
Iniciado supervisor s1
Iniciado supervisor s2
Iniciado worker w2A 
Iniciado worker w2B 
Iniciado supervisor s3
Iniciado worker w3A 
Iniciado supervisor s5
Iniciado worker w5A 
Iniciado worker w5B 
Iniciado worker w5C 
Iniciado supervisor s4
Iniciado worker w4A 
Iniciado worker w4B 
ok
\end{mycode}
\fi

\end{document}
